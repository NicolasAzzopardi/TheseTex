\begin{vcenterpage}
\chapter*{Résumé}
%\markright{\MakeUppercase{Résumé}}
Les anticorps monoclonaux ont révolutionné le traitement de nombreuses pathologies. Cependant, leur pharmacocinétique (PK) et l'influence de leur concentration sur la réponse clinique restent mal connues. Nous avons étudié les sources de variabilité interindividuelle de la PK du cetuximab, un anticorps anti-EGFR, ainsi que l'influence de l'exposition à cet anticorps sur la réponse. Nous avons validé une méthode ELISA de dosage du cetuximab. Dans un modèle murin, nous avons étudié l'absorption pulmonaire du cetuximab. Nous avons étudié la PK du cetuximab chez un patient hémodialysé. Nous avons décrit la PK du cetuximab chez des patients traités pour cancer colorectal métastatique, à l'aide d'un modèle combinant des éliminations d'ordre 0 et 1. Enfin, nous avons identifié la clairance globale du cetuximab, paramètre pouvant être estimé précocement par la concentration résiduelle à J14, comme un facteur influençant la survie sans progression des patients. Nos travaux montrent qu'une description de la PK d'un anticorps par approche compartimentale permet d'identifier les sources de variabilité et d'étudier l'impact de la PK sur la réponse clinique.
\paragraph*{Mots clés :} anticorps monoclonaux, cetuximab, ELISA, pharmacocinétique, relation dose-réponse, survie sans progression, polymorphisme génétique, récepteurs Fc
\end{vcenterpage}