\begin{vcenterpage}

\chapter*{Abstract}
%\markright{\MakeUppercase{Abstract}}
Monoclonal antibodies have profoundly modified the treatment of many diseases. However, their pharmacokinetics (PK) and the influence of their concentrations on the clinical response are poorly known. We studied the sources of the interindividual variability of PK of cetuximab, an anti-EGFR, and the influence of the exposure to this antibody on the response. We validated an ELISA technique to measure cetuximab concentrations. We studied the pulmonary absorption of cetuximab in a murine model. We studied cetuximab PK in a hemodialysed patient. In metastatic colorectal cancer patients, we described cetuximab PK with the help of a model combining zero- and first-order eliminations. Finally, we identified the global clearance of cetuximab, a parameter which can be estimated by residual concentration on day 14, as a factor influencing progression-free survival of the patients. Our work shows that the description of the PK of an antibody by compartmental approach allows to identify sources of variability and to study the impact of PK on the clinical response.
\paragraph*{Keywords :} monoclonal antibodies, cetuximab, ELISA, pharmacokinetics, dose-response relationship, genetic polymorphism, Fc receptors

\end{vcenterpage}