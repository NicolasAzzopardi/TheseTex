\thispagestyle{empty}
\setmarginsrb{20mm}{20mm}{25mm}{0mm}{0mm}{0mm}{0mm}{0mm}
\begin{tabular}{ p{3cm} p{9cm} p{3cm}}
		\begin{minipage}{3cm}
			\includegraphics[width=3cm]{./images/pucvl.png} 
		\end{minipage}
	&
		\begin{minipage}{9cm}
			\begin{center}
				\textbf{Apport de la modélisation pharmacocinétique à l'étude de la variabilité de réponse aux anticorps monoclonaux antitumoraux : Application au cetuximab}

Nicolas AZZOPARDI
			\end{center}
		\end{minipage}
	&
		\begin{minipage}{3cm}
			\includegraphics[width=3cm]{./images/ufr.png} 
		\end{minipage}
\end{tabular}
\vspace{1cm}
\begin{minipage}{16cm}
\section*{Résumé :}
Les anticorps monoclonaux ont révolutionné le traitement de nombreuses pathologies. Cependant, leur pharmacocinétique (PK) et l'influence de leur concentration sur la réponse clinique restent mal connues. Nous avons étudié les sources de variabilité interindividuelle de la PK du cetuximab, un anticorps anti-EGFR, ainsi que l'influence de l'exposition à cet anticorps sur la réponse. Nous avons validé une méthode ELISA de dosage du cetuximab. Dans un modèle murin, nous avons étudié l'absorption pulmonaire du cetuximab. Nous avons étudié la PK du cetuximab chez un patient hémodialysé. Nous avons décrit la PK du cetuximab chez des patients traités pour cancer colorectal métastatique, à l'aide d'un modèle combinant des éliminations d'ordre~0 et 1. Enfin, nous avons identifié la clairance globale du cetuximab, paramètre pouvant être estimé précocement par la concentration résiduelle à J14, comme un facteur influençant la survie sans progression des patients. Nos travaux montrent qu'une description de la PK d'un anticorps par approche compartimentale permet d'identifier les sources de variabilité et d'étudier l'impact de la PK sur la réponse clinique.
\paragraph*{Mots clés :} anticorps monoclonaux, cetuximab, ELISA, pharmacocinétique, relation dose-réponse, survie sans progression, polymorphisme génétique, récepteurs Fc
\vspace{1cm}
\section*{Abstract :}
Monoclonal antibodies have profoundly modified the treatment of many diseases. However, their pharmacokinetics (PK) and the influence of their concentrations on the clinical response are poorly known. We studied the sources of the interindividual variability of PK of cetuximab, an anti-EGFR, and the influence of the exposure to this antibody on the response. We validated an ELISA technique to measure cetuximab concentrations. We studied the pulmonary absorption of cetuximab in a murine model. We studied cetuximab PK in a hemodialysed patient. In metastatic colorectal cancer patients, we described cetuximab PK with the help of a model combining zero- and first-order eliminations. Finally, we identified the global clearance of cetuximab, a parameter which can be estimated by residual concentration on day 14, as a factor influencing progression-free survival of the patients. Our work shows that the description of the PK of an antibody by compartmental approach allows to identify sources of variability and to study the impact of PK on the clinical response.
\paragraph*{Keywords :} monoclonal antibodies, cetuximab, ELISA, pharmacokinetics, dose-response relationship, genetic polymorphism, Fc receptors
\end{minipage}