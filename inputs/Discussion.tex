\chapter{Discussion}
Le développement et la validation de la méthode de dosage de cetuximab par technique ELISA ont permis d'obtenir des mesures fiables des concentrations sériques de cetuximab. Un certain nombre de techniques de dosage du cetuximab dans le sérum sont mentionnées dans la littérature mais aucune n'est décrite de façon détaillée. Il ne nous est donc pas possible de comparer les performances de notre technique avec celles rapportées par d'autres équipes. L'utilisation d'une protéine recombinante produite dans le cadre d'une collaboration régionale implique que nos résultats devront, à terme, être comparés avec ceux obtenus par d'autres techniques ELISA. Les concentrations cibles données dans la publication~IV sont donc à interpréter avec prudence si une autre technique est utilisée. Toutefois, ceci ne modifie pas la conclusion principale de ce travail, qui est que le suivi thérapeutique des concentrations sériques de cetuximab chez les patients traités pour cancer colorectal métastatique pourrait être utile. 

De façon générale, l'erreur résiduelle d'un modèle pharmacocinétique de population décrivant des concentrations mesurées par une méthode ELISA est décrite de façon satisfaisante par un modèle mixte additif et proportionnel~\citep{REF68, REF73}. La partie additive du modèle prend en compte le bruit de fond de la technique, tandis que la partie proportionnelle suit la portion linéaire croissante de la courbe de calibration. En deçà de la limite basse de quantification de la technique (LLOQ), il est recommandé de censurer la valeur de la concentration, pour permettre une meilleur estimation de l'élimination à concentration basse~\citep{REF142}. Cette censure, bien que justifiée pour des raisons analytiques, peut poser problème lors du choix du modèle structural~\citep{REF143}. Dans l'étude pharmacocinétique de la publication~III, qui prenait en compte la censure des concentrations inférieures à la LLOQ, un problème d'identifiabilité entre la variabilité interindividuelle de $k_0$ et la partie additive du modèle d'erreur résiduelle a été rencontré : la variabilité de l'erreur résiduelle additive était très proche de zéro et la variabilité de $k_0$ présentait un important "shrinkage" (variabilité interindividuelle estimée trop élevée). Le problème a été contourné en utilisant un modèle d'erreur résiduelle proportionnel (équation~\ref{eq:51}). La variabilité interindividuelle de $k_0$ correspondait donc également au « bruit de fond » de la technique analytique. Dans les études publiées, les modèles d'erreur utilisés pour décrire l'élimination des anti-EGFR étaient de type mixte additif et proportionnel~\citep{REF68, REF73}. Cependant, les concentrations inférieures aux limites de quantification des techniques de dosage utilisées n'étaient pas censurées. Dans l'étude pharmacocinétique de la publication~III, les concentrations basses étaient très informatives pour la quantification de l'élimination saturable. Supprimer les concentrations inférieures à la LLOQ n'est pas recommandé ; ceci serait une perte d'information~\citep{REF142}. Le problème d'identifiabilité rencontré ici pourrait sans doute être résolu en adaptant le modèle d'erreur résiduelle au profil de la courbe de calibration de la technique de dosage. Dans ce modèle, les concentrations proches des limites basse et haute de quantification seraient prises en compte par une pondération moins importante que celle des concentrations issues de la partie linéaire de la courbe de calibration.

Les résultats de la modélisation individuelle du cetuximab réalisée chez un patient insuffisant rénal dialysé ont montré que la pharmacocinétique de cet anticorps était comparable à celle décrite chez des patients sans insuffisance rénale. La pharmacocinétique du cetuximab ne semble donc être influencée ni par un dysfonctionnement rénal ni par une dialyse chronique, bien que ces résultats doivent être confirmés sur un plus grand nombre de sujets. Les injections étaient trop rapprochées pour identifier une élimination saturable chez ce patient, bien que les concentrations observées entre le 49ème et le 56ème jour aient un profil comparable à celle des patients de la publication~III.

La publication~III confirme la non-linéarité de la pharmacocinétique du cetuximab décrite par Dirks~\textit{et al.}, bien qu'ils aient décrit l'ensemble de l'élimination comme étant de type Michaelis-Menten~\citep{REF68}. Notre modèle semble plus mécanistique car il permet la description des deux types d'éliminations connues pour les anticorps thérapeutiques. Cette approche a déjà été appliquée par Ma~\textit{et al.} au panitumumab~\citep{REF73} et par Kuester~\textit{et al.} au matuzumab~\citep{REF129}. Dans notre travail, l'élimination non saturable ($\CL$) était reliée à l'albuminémie, ce qui pourrait correspondre à l'implication du FcRn dans ce mode d'élimination. En effet, un FcRn avec une moins bonne affinité pour la portion Fc des anticorps et pour l'albumine et$/$ou un FcRn présent en plus petite quantité va recycler une plus faible quantité d'anticorps et d'albumine. La clairance de ces deux protéines sera donc augmentée. Le paramètre $k_0$ étant relié à la surface corporelle (BSA) et à la toxicité cutanée, l'élimination saturable pourrait quant à elle décrire l'élimination spécifique du cetuximab, consécutive à sa fixation sur l'EGFR, présent en quantité importante sur les épithéliums. Une publication récente présente des résultats in vitro de mesure de la constante de dissociation $K_d = k_{off}/k_{on}$ de trois anticorps anti-EGFR (cetuximab, panitumumab et matuzumab)~\citep{REF144}. Cette étude rapporte des $K_d$ très faibles pour les trois anticorps, ce qui justifie une simplification d'un modèle de type TMDD en un modèle intégrant une élimination de type Michaelis-Menten ou même une élimination d'ordre~0 ($k_0$) pour décrire l'élimination spécifique d'un anticorps anti-EGFR. Cependant, cette simplification ne permet pas de prendre en compte les variations de quantité de cible de l'anticorps au cours du temps. L'explication de ces variations est pourtant indispensable à une bonne description de l'élimination spécifique de l'anticorps et donc à la description de sa relation concentration-effet. Dans le cadre de cette thèse, toutes les tentatives de description de la pharmacocinétique du cetuximab (publication~III) par des modèles plus complexes (approches TMDD notamment), prenant en compte une variation de la quantité de cible, étaient tous sur-paramétrés.

Compte tenu de cette non linéarité, une modification de la posologie telle qu'une augmentation de l'intervalle de temps entre les injections doit reposer sur une simulation pharmacocinétique. Nous avons utilisé le modèle pharmacocinétique décrit dans la publication~III pour simuler les concentrations attendues avec différentes doses de perfusion pour des intervalles d'une, deux et trois semaines (manuscrit~IV). Les résultats montrent que le doublement ou le triplement de la dose sont insuffisants pour maintenir des concentrations résiduelles comparables lorsque l'intervalle entre deux perfusions est doublé ou triplé. Bien que significativement différente en raison du grand nombre de patients simulés, la distribution des valeurs d'$\AUC$ pour les posologies ayant la même dose cumulée sur trois semaine mais des intervalles d'injection différents, ne montre pas de différences d'échelle aussi importante que les concentrations résiduelles. Néanmoins, ces résultats préliminaires justifient un suivi minutieux des concentrations de cetuximab dans les études testant un espacement des perfusions.  

Pour réaliser une modélisation de type TMDD du cetuximab et de l'EGFR, il faudrait disposer d'une estimation des quantités d'EGFR libres et/ou liées au cetuximab, mais également de la proportion d'EGFR présente sur les cellules cancéreuses, et sur les cellules saines (épithéliums et endothéliums entre autres). Cependant, la difficulté de quantification de l'EGFR \textit{in vivo} et sa double localisation sur les tissus tumoraux et sains rendent cette modélisation difficile. Le critère RECIST est un indicateur efficace de la réponse d'un patient, mais c'est un critère composite qui présente une forte variabilité interindividuelle. La description de son évolution par approche PK-PD s'est révélée impossible. Une modélisation reposant sur un modèle de type TMDD aurait eu pour conséquences une sur-paramétrisation compte tenu des données disponibles. L'utilisation d'autres biomarqueurs tels que le CA19.9 ou l'ACE présentaient les mêmes types de limitations. 

De plus, dans l'étude de la publication~III, les doses de chimiothérapie associée étaient ajustées individuellement. Cette influence a été recherchée sans succès dans les essais de modélisation PK-PD du cetuximab et dans l'étude de la survie sans progression, par le biais de la dose cumulée ou du nombre d'injections. Cependant, il semble logique de penser que l'effet de cette chimiothérapie ne doit pas être séparé de la relation dose-concentration-effet du cetuximab dans le cancer colorectal. Il faudrait donc modéliser de façon conjointe la relation dose-concentration-effet du cetuximab, de l'irinotecan et du 5-FU pour quantifier l'influence de l'exposition à ces trois médicaments. Ce modèle entièrement paramétrique pourrait permettre de comparer par simulation l'intérêt de différentes posologies d'irinotecan, de 5-FU et de cetuximab, lors de leur association. 

Les résultats de la publication~III montrent que la survie sans progression (PFS) n'est pas influencée par l'élimination saturable ou l'élimination non-saturable du cetuximab, mais par la clairance globale estimée sur le temps avant progression, c'est-à-dire, la combinaison de $\CL$ et $k_0$. Ce ne sont donc pas les phénomènes d'élimination mais l'exposition au cetuximab qui en résulte qui influence la PFS des patients traités.

Une bonne estimation de la clairance globale nécessite une étude de la pharmacocinétique pendant un temps assez long. Si cela est possible rétrospectivement à partir des données d'une étude clinique comme celle-ci, cela n'est pas réalisable couramment en clinique. Les travaux de la publication~III montrent que la concentration résiduelle de cetuximab à J14 est un bon reflet de la clairance globale et est reliée à la survie sans progression. Cette concentration résiduelle à J14 pourrait donc être utilisée en clinique comme indicateur précoce de la clairance globale et donc de la survie sans progression. Après confirmation par une étude prospective, les patients ayant une concentration résiduelle trop basse pourraient alors bénéficier d'un ajustement précoce de dose pour optimiser leur réponse clinique.

Une relation entre la concentration résiduelle et la réponse a été rapportée pour un autre anticorps monoclonal antitumoral indiqué dans les tumeurs solides, le trastuzumab~\citep{REF145} Notre approche pourrait donc s'appliquer à d'autres anticorps antitumoraux.

La modélisation de la pharmacocinétique du cetuximab administré par voie pulmonaire chez la Souris a permis la description de son absorption. Cette faible biodisponibilité (F proche de 10 $\%$) est un élément positif car il montre une faible exposition systémique (donc un faible risque d'effets indésirables tels que les effets cutanés) et devrait correspondre à une forte exposition loco-régionale (pulmonaire). Cette accumulation du cetuximab dans le poumon est confirmée par les données d'imagerie obtenues avec le cetuximab marqué au Xenofluor. Ces résultats encourageants ont incité l'équipe INSERM U618 à lancer une nouvelle étude équivalente chez le Singe. Cette étude est actuellement en cours d'analyse.

Notre travail montre que, comme les autres anticorps thérapeutiques, le cetuximab a une grande variabilité pharmacocinétique interindividuelle. Ceci justifie la recherche des sources de cette variabilité pharmacocinétique et nécessite d'étudier l'impact de cette variabilité sur la réponse au traitement, c'est-à-dire la relation concentration-effet. Ces deux domaines d'étude reposent sur une description quantitative précise de la pharmacocinétique, ce qui est un défi compte tenu de la complexité des phénomènes impliqués dans le devenir des anticorps dans l'organisme. Nous avons pu appliquer avec succès un modèle bicompartimental avec ou sans absorption d'ordre~1 au modèle murin, un modèle bicompartimental "simple" au patient en insuffisance rénale traité par hémodialyse et un modèle avec élimination mixte (non saturable et saturée) aux patients traités pour cancer colorectal métastatique. Ce dernier modèle nous a permis d'identifier des facteurs de variabilité pharmacocinétique associés au traitement de cette pathologie. Il semble cependant nécessaire de développer des modèles plus complexes, faisant intervenir éventuellement la quantité d'antigène-cible et/ou l'influence des  médicaments associés, pour identifier plus précisément l'ensemble des facteurs de variabilité pharmacocinétique. Le modèle obtenu pourrait permettre de mieux prédire la pharmacocinétique du cetuximab et ainsi d'argumenter de façon plus pertinente une optimisation individuelle de sa posologie. Un modèle pharmacocinétique mieux ajusté aux données pourrait également permettre d'améliorer l'étude de la relation concentration-réponse et donc de permettre, si un biomarqueur pertinent est disponible, d'étudier les sources individuelles de variabilité de sensibilité au traitement, tels que les polymorphismes des récepteurs Fc$\gamma$R.
