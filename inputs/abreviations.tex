%\chapter*{Abréviations}
\printglossaries

%\addcontentsline{toc}{chapter}{Abréviations}
%\markright{\MakeUppercase{Abréviations}}
\begin{tabularx}{30em}{X X}
5-FU & 5-fluorouracile\\
$\alpha$ & 	Pente de décroissance de la phase de distribution d'un modèle bi-compartimental\\
AMM & 	Autorisation de mise sur le marché\\
$\gls{AUC}_n$ & 	AUC normalisée à la dose\\
ATI & 	Anticorps anti-infliximab (\textit{Antibodies toward infliximab})\\
$\beta$ & 	Pente de décroissance de la phase d'élimination d'un modèle bi-compartimental\\
$C$ & 	Concentration de médicament\\
$C_{max}$ & 	Concentration maximale du médicament observée après une dose unique\\
$\CL$ & 	Clairance systémique\\
$C_n$ & 	Dernière concentration mesurée\\
CRP & 	Protéine C-réactive (\textit{C-reactive protein})\\
$D$ & 	Dose\\
Da & 	Dalton : Unité de masse atomique $\approx  1,66054\times 10^{-27} kg$\\
DPD & 	Dihydropyrimidine déshydrogénase\\
EGF & 	Facteur de croissance épidermique (\textit{Epidermal growth factor})\\
EGFR & 	Récepteur au facteur de croissance épidermique. Aussi appelé HER1 ou c-ErbB-1 (\textit{Epidermal growth factor receptor})\\
ELISA & 	\textit{Enzyme-linked immunosorbent assay}\\
ERK & 	\textit{Extracellular signal-regulated kinase}\\
EV & 	Extra-vasculaire (administration)\\
$F$ & 	Fraction absorbée après administration extravasculaire\\
Fc$\gamma$RIIIA & 	Récepteur membranaire de la portion Fc des immunoglobulines G. Aussi appelé CD16a\\
\textit{FCGR3A} & 	Gène codant Fc$\gamma$RIIIA\\
FDA & 	Food and Drug Administration\\
Ig & 	Immunoglobuline. Chez l'Homme, elles sont de classe A, D, E, G ou M : IgA, IgD, IgE, IgG ou IgM \\
IV & 	Intra-veineuse (administration)\\
JAK2 & 	Janus Kinase 2\\
$k_{10}$ & 	Constante d'élimination d'ordre~1 également notée $k_e$\\
$k_{12}$ & 	Constante de transfert d'ordre~1 du compartiment~1 vers le compartiment~2\\
$k_{21}$ & 	Constante de transfert d'ordre~1 du compartiment~2 vers le compartiment~1\\
$k_a$ & 	Constante d'absorption d'ordre~1 également notée $k_{01}$\\
$k_t$ & 	Constante de transfert d'ordre~1 entre compartiments de transit\\
LC-MS/MS & 	Chromatographie en phase liquide couplée à une spectrométrie de masse en tandem (\textit{Liquid chromatography coupled to tandem mass spectrometry})\\
$\mu$ & 	Moyenne\\
$\MAT$ & 	Temps d'absorption moyen (\textit{Mean absorption time})\\
MAPK & 	Kinases activées par des agents mitogènes (\textit{Mitogen-activated protein kinase})\\
MM & 	Michaelis-Menten\\
$\MRT$ & 	Temps de résidence moyen (\textit{Mean residence time})\\
NPDE & 	Distribution normalisée des résidus (\textit{Normalized prediction distribution error})\\
OS & 	Survie globale (Overall survival)\\
PFS & 	Survie sans progression (\textit{Progression-free survival})\\
PI3K & 	Phosphatidylinositol 3-kinase (Phosphoinositide 3-kinase)\\
PK & 	Pharmacocinétique (\textit{Pharmacokinetics})\\
PK-PD & 	Relation pharmacocinétique-pharmacodynamique (\textit{Pharmacokinetic-Pharmacodynamic relationship})\\
$Q$ & 	Clairance de distribution, également notée $CL_2$\\
$\sigma$ & 	Écart-type\\
SC & 	Sous-cutanée (administration)\\
STAT3 & 	\textit{Signal transducer and activator of transcription 3}\\
$\tau$ & 	Intervalle de temps entre deux administrations\\
$T$ & 	Durée de la perfusion\\
TGF$\alpha$ & 	Facteur de croissance transformant alpha (\textit{Transforming growth factor-alpha})\\
TK & 	Tyrosine kinase \\
$T_{max}$ & 	Temps correspondant à la concentration maximale $C_{max}$ dans le cas d'une administration unique\\
TMDD & 	Élimination liée à la cible (\textit{Target-mediated drug disposition})\\
TV & 	Valeur typique d'un paramètre (\textit{Typical value})\\
$t_{1/2}$ & 	Demi-vie dans le cas d'un modèle bi-compartimental ; $t_{1/2}\alpha$ = demi-vie de distribution ; $t_{1/2}\beta$ = demi-vie d'élimination.\\
$V$ & 	Volume de distribution ($V_{SS}$, $V_1$, $V_2$)\\
\end{tabularx}
