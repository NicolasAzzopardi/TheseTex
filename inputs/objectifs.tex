\chapter{Objectifs}
%\addcontentsline{toc}{chapter}{Objectifs}
%\markright{\MakeUppercase{Objectifs}}

L'objectif de notre travail était d'étudier la pharmacocinétique du cetuximab afin d'identifier ses sources de variabilité interindividuelles, d'étudier son influence sur la réponse clinique et d'explorer de nouveaux modes d'administration. Pour répondre à cet objectif, les travaux suivants ont été réalisés :


\begin{itemize}
\item Étude de la pharmacocinétique du cetuximab durant la dialyse rénale (publication~I). Le dysfonctionnement d'organes peut entraîner une modification de la pharmacocinétique des médicaments conventionnels. Peu de données sont disponibles sur la pharmacocinétique des anticorps monoclonaux chez les patients avec insuffisance rénale terminale traités par dialyse.


\item Étude des facteurs influençant la pharmacocinétique du cetuximab dans le cancer colorectal métastatique (Publications~II~et~III). Bien qu'il soit indiqué dans le traitement du cancer colorectal métastatique, la pharmacocinétique du cetuximab n'avait jamais été étudiée de façon précise dans cette pathologie. Notre étude a eu pour but de décrire la relation entre la dose et la concentration de cetuximab au cours du temps, dans le cancer colorectal métastatique par approche de population afin d'identifier ses sources de variabilité.


\item Étude de l'influence de la pharmacocinétique sur la réponse au cetuximab dans le cancer colorectal métastatique (publication~III). Grâce à l'étude décrite ci-dessus, nous avons étudié la relation entre la pharmacocinétique du cetuximab et la survie sans progression.


\item Étude de l'espacement des perfusions sur l'exposition des patients (manuscrit~IV). Grâce au modèle de l'étude~IV, nous avons simulé les concentrations obtenues avec différentes doses de cetuximab administrées toutes les une, deux ou trois semaines.


\item Étude de l'exposition au cetuximab après administration pulmonaire (publication~V). La nébulisation est une voie d'administration~EV prometteuse pour le traitement du cancer broncho-pulmonaire. Nous avons étudié la biodisponibilité du cetuximab administré par nébulisation chez la Souris, étape indispensable avant une éventuelle étude clinique.
\end{itemize}

